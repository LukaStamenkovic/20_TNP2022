% !TEX encoding = UTF-8 Unicode

\documentclass[a4paper]{article}

\usepackage{color}
\usepackage{url}
\usepackage[T2A]{fontenc} % enable Cyrillic fonts
\usepackage[utf8]{inputenc} % make weird characters work
\usepackage{graphicx}
\usepackage[document]{ragged2e}
\usepackage{}
\usepackage[english,serbian]{babel}
\usepackage[
    style=numeric,
    sorting=none
]{biblatex}
\addbibresource{seminarski.bib}

\usepackage[unicode]{hyperref}
\hypersetup{colorlinks,citecolor=green,filecolor=green,linkcolor=blue,urlcolor=blue}



\begin{document}
\title{Svemir kao velika neuronska mreža\\ \small{Seminarski rad u okviru kursa\\Tehničko i naučno pisanje\\ Matematički fakultet}}

\author{Aleksa Filipović, Marta Kiso, Veljko Seničanin, Luka Stamenković\\stamenkovicl777@gmail.com}
\date{15.~novembar 2022.}
\maketitle
\abstract{\justifying{
U ovom radu ćemo diskutovati o mogućnosti da je ceo univerzum jedna velika neuronska mreža. Samu ideju je dao ruski naučnik Vitali Vančurin sa Univerziteta u Minesoti u svom naučnom radu. \cite{1}
\\Princip kog ćemo pokušati da se držimo jeste taj da ne zalazimo previše u oblasti fizike i astronomije, već da na što jednostavniji način pokušamo da približimo ovu temu čitaocima. Opisaćemo šta je to neuronska mreža, kao i da li i na koji način je možemo povezati sa kosmososom.
}}
\tableofcontents

\newpage

\section{Uvod}
\label{sec:uvod}
\justifying
Kako bismo uopšte mogli da razmatramo opciju da se svemir može zamisliti kao neuronska mreža, važno je da tačno definišemo šta je to neuronska mreža. U uvodu ćemo je posmatrati iz biološkog ugla, dok ćemo u narednom poglavlju diskutovati o neuronskoj mreži u vidu implementacije sistema veštačke inteligencije. Neuronska mreža predstavlja nervni sistem živih bića, u bukvalnom smislu je to jedan veliki skup povezanih nervnih ćelija - neurona.  

Astronomi su pokušali da utvrde razlike i sličnosti dva najsloženija nama poznata sistema, mada u potpuno različitim razmerama – svemira i njegovih galaksija i mozga i njegovih neuronskih ćelija. Ispostavilo se da su sličnosti mnogo veće nego što se moglo pretpostaviti. Okupljeni u timu, astorfizičari, neurolozi i neurohirurzi su otkrili da su, i pored toga što su razmere neuporedive, strukture mozga i svemira izuzetno slične. U nekim aspektima, ova dva sistema izgledaju sličnija jedan drugom, nego delovima koji ih čine.  

Navodi se da izuzetno različiti fizički procesi dovode do vrlo sličnih složenih i organizovanih struktura. Na primer, ljudski mozak deluje zahvaljujući mreži od skoro 70 milijardi neurona koji ga zajedno čine. Analogno, pretpostavlja se da svemir ima najmanje 100 milijardi galaksija.\cite{6} Oba ova sistema su organizovana u složenu mrežu povezanu dugim nitima i čvorovima koji ih povezuju Ovi čvorovi se uočavaju na slikama i svemira i mozga, što objašnjava neke očigledne sličnosti. Konkretno, na slici \ref{fig:komparacija} možemo vrlo lako uvideti sličnost.

\begin{figure}[h!]
\begin{center}
\includegraphics[scale=0.2]{komparacija.jpeg}
\end{center}
\caption{Komparacija ćelija mozga i galaksija}
\label{fig:komparacija}
\end{figure}

\justifying
Takođe, u oba sistema, ove mreže čine samo 30 odsto mase. U svakom od njih oko 70 odsto mase zapravo čine delovi za koje se čini da su pasivni: moždana tečnost i tamna materija svemira.

Vitali Vančurin misli da ako posmatramo svemir kao neuronsku mrežu, njegovo ponašanje pod određenim uslovima možemo objasniti jednačinama kvantne mehanike i zakonima klasične fizike, poput teorije relativnosti koju je osmislio Albert Ajnštajn. Vačurin smatra da bi daljim proučavanjem ove teorije mogao rešiti glavni problem moderne fizike \--\ neslaganje klasične mehanike,
koja opisuje kako svemir funkcioniše u velikim razmerima i kvantne mehanike koja se bavi proučavanjem atomskog i subatomskog nivoa materije.  Kao što je poznato, razlika između klasične i kvantne mehanike jeste ta što je u klasičnoj mehanici vreme apsolutno i univerzalno, dok je u kvantnoj mehanici relativno.

 Vančurin je u intervjuu za časopis \textit{Futurism} \cite{7} istakao ključnu tačku svog rada. Naime, on je zaključio da je dinamika učenja neuronskih mreža veoma slična kvantnoj dinamici u fizici. Ono što je zapravo rečeno je da je učenje neuronskih mreža u nekim granicama logična, dok izvan tih granica ne važe ista pravila kao unutar njih. To se može uporediti sa zakonima klasične i kvantne mehanike. Najbolje bi to bilo predstaviti u tabeli \ref{tab:tabela1}
 
\begin{table}[h!]
\begin{center}
\rowcolors{2}{gray!10}{gray!40}
\begin{tabular}{|c|c|} \hline
Klasična fizika& Kvantna fizika\\ \hline
Telo poseduje ili čestična ili talasna svojstva & Telo poseduje i čestična i talasna svojstva\\ 
Možemo tačno odrediti i poziciju i brzinu tela &Nemoguće je tačno odrediti\\ 
Ne postoji dilatacija vremena& Postoji dilatacija vremena \\ 
Ne postoji kontrakcija dužine& Postoji kontrakcija dužine\\ \hline
\end{tabular}
\caption{Razlike između klasične i kvantne fizike}
\label{tab:tabela1}
\end{center}
\end{table}

\section{Neuronske mreže}
\label{sec:naslov1}

Neuronske mreže (eng. {\textit{neural networks}}) predstavljaju najpopularniju i jednu od najprimenjenijih metoda mašinskog učenja. Njihove primene su mnogobrojne i pomeraju domete veštačke inteligencije, računarstva i primenjene matematike. Neke od njih su:
\begin{itemize}
  \item kategorijzacija teksta
  \item medicinska dijagnostika
  \item prepoznavanje objekata na slikama
  \item autonomna vožnja
  \item igranje igara poput igara na tabli (tavla i go) ili video igara
  \item mašinsko prevođenje prirodnih jezika
  \item modelovanje semantike reči prirodnog jezika i slično
\end{itemize}
Neuronske mreže zapravo predstavljaju parametrizovanu reprezentaciju koja može poslužiti za aproksimaciju drugih funkcija. Kao i u slučaju drugih metoda učenja, pronalaženje odgovarajućih parametara se vrši matematičkom optimizacijom nekog kriterijuma kvaliteta aproksimacije i može biti računski vrlo izazovno.

\begin{figure}[h!]
\begin{center}
\includegraphics[scale=1]{fullyconnected.jpg}
\end{center}
\caption{Struktura potpuno povezane neuronske mreže.}
\label{fig:fullyconnected}
\end{figure}

Postoje različite vrste neuronskih mreža. Osnovnu varijantu predstavljaju potpuno povezane neuronske mreže (eng. {\textit{fully connected}}), koju možemo videti na slici \ref{fig:fullyconnected}. U obradi slika i drugih vrsta signala, pa i teksta, vrlo su popularne konvolutivne neuronske mreže (eng. {\textit{convolutional neural networks}}). Za obradu podataka nalik nizovima promenljive dužine, najčešće se koriste rekurentne neuronske mreže (eng. {\textit{recurrent nerual networks}}), a za obradu podataka koji se predstavljaju grafovima koriste se grafovske neuronske mreže (eng. {\textit{graph neural networks}}).

U svetlu njihovih izvanrednih uspeha i velike popularnosti, u laičkim krugovima postoji tendencija poistovećivanja mašinskog učenja, pa čak i veštačke inteligencije sa neuronskim mrežama. Ovakav pogled je prosto pogrešan. Takođe, postoji tendencija da se neuronska mreža razmatra kao prvi izbor metoda učenja nevezano od toga o kom se problemu radi. Ovo bi bio vrlo loš praktičan savet. Stoga naglašavamo u kakvim situacijama su superiorne u odnosu na druge modele. Dok je sasvim moguće da će neuronske mreže prestići druge modele i u drugačijim problemima, problemi zahvaljujući kojima su se neuronske mreže proslavile imaju određena zajednička svojstva. To su velika količina podataka i učenje na osnovu sirove reprezentacije podataka. Male količine podataka u slučaju neuronskih mreža lako vode preprilagođavanju neuronskih mreža, odnosno sposobnosti da same konstruišu nove atribute nad sirovom reprezentacijom podataka. Neuronske mreže su karakteristične po tome što su u stanju to da rade. Stoga, ukoliko je skup podataka mali nema razloga da očekujemo posebni benefit od primene neuronskih mreža, a moguće je da ćemo imati problema sa njihovim nedostacima. Ukoliko je veliki i u sirovom obliku, verovatno je dobra ideja primeniti neku varijaciju neuronske mreže.

\printbibliography[
heading=bibintoc,
title={Literatura}
]

\end{document}

\section{Vančurijev rad}

Godinama su fizičari pokušavali da pomire kvantnu mehaniku i opštu teoriju relativiteta. Kvantna mehanika tvrdi da je vreme apsolutno i univerzalno dok opšta teorija relativnosti tvrdi da je vreme relativno.

Vančurin tvrdi da neuronske mreže mogu da pokažu "približno ponašanje" obe teorije. Pošto je kvantna mehanika "izuzetno uspešna ideologija za modeliranje fizičkih pojava na širokom spektru skala", rasprostranjeno je verovanje da je na najosnovnijem nivou čitav univerzum vođen pravilima kvantne mehanike, pa da čak i gravitacija treba da izađe iz toga. On ne kaže samo da veštačke neuronske mreže mogu biti korisne za analizu fizičkih sistema ili za otkrivanje fizičkih zakona, već i da svet oko nas zapravo funkcioniše tako. U tom smislu to bi se moglo smatrati predlogom teorije svega, i kao takvo trebalo bi se dokazati da je pogrešno. 

Kako bi na jednostavniji način objasnio svoju teoriju, Vančurin je pokušao na dva načina pojasni svoje zaključke. Prvi način je da se počne sa preciznim modelom neuronskih mreža, a zatim da se proučava ponašanje mreže u granicama velikog broja neurona. Ono što je pokazao je to da jednačine kvantne mehanike prilično dobro opisuju ponašanje sistema blizu ravnoteže, a jednačine klasične mehanike opisuju kako je sistem dalje od ravnoteže. Drugi način je da krenemo od fizike. Znamo da kvantna mehanika radi jako dobro na malim malim, a opšta teorija relativiteta na velikim razmerama, no svakako nisu uspeli da pomire te dve velike teorije. Ovo je poznato kao pojam kvantne gravitacije. No to nije ni jedini problem, kako navodi Vančurin, već je i jedan od većih problema - problem posmatrača. S tim bi se moglo tvrditi da postoje tri fenomena koja treba objeniti. Većina fizičara bi reklo da je osnova kvantna mehanika i da nekako sve ostalo mora nastati iz nje, ali niko ne  zna kako. 

Vitali je takođe sagledao mogućnost da je mikroskopska neuronska mreža osnovna struktura i da sve ostalo proizilazi iz nje. Do ideje je došao u jednom od svojih starih radova \cite{5}. Prvobitna ideja je bila da se metode statističke mehanike primene za proučavanje ponašanja neuronskih mreža, ali se pokazalo da je u određenim granicama dinamika učenja neuronskih mreža veoma slična kvantnoj dinamici koju vidimo u fizici. Tada je došao na ideju da je fizički svet zapravo neuronska mreža. Vitali tvrdi da, kako bi se dokazalo da je ova teorija pogrešna, treba naći samo jedan fizički fenomen koji se ne može opisati neuronskim mrežama. Ovo je , naravno jako teško, jer ljudi jako malo znaju o tome kako se neuronske mreže ponašaju i kako mašinsko učenje zapravo funkcioniše. 

Što se tiče kvantne mehanike on navodi kako postoje dve linije mišljenja a to su Everetova interpretacija i Bomovo tumačenje. Vančurin nema ništa da kaže o tumačenju paralelnih univerzuma, ali navodi da može da doprinese teorijama skrivenih promenljivih. Skrivene promenljive su stanja pojedinačnih neurona, a promenljive koje se mogu obučiti su kvantne promenljive. Skrivene promenljive mogu biti nelokalne i time mogu narušiti Belove nejednakosti.
